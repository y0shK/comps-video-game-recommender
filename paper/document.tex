\documentclass[10pt,twocolumn]{article}

% use the oxycomps style file
\usepackage{oxycomps}

% usage: \fixme[comments describing issue]{text to be fixed}
% define \fixme as not doing anything special
\newcommand{\fixme}[2][]{#2}
% overwrite it so it shows up as red
\renewcommand{\fixme}[2][]{\textcolor{red}{#2}}
% overwrite it again so related text shows as footnotes
%\renewcommand{\fixme}[2][]{\textcolor{red}{#2\footnote{#1}}}

% read references.bib for the bibtex data
\bibliography{references}

% include metadata in the generated pdf file
\pdfinfo{
    /Title (Reverse Engineering the GiantBomb API: Using Word Embeddings to Predict Video Game Recommendations)
    /Author (Yash Karandikar)
}

% set the title and author information
\title{Reverse Engineering the GiantBomb API: Using Word Embeddings to Predict Video Game Recommendations}
\author{Yash Karandikar}
\affiliation{Occidental College}
\email{ykarandikar@oxy.edu}

\begin{document}

\maketitle

\section{Introduction and Problem Context}

%This section should motivate why the project is interesting both to you and to the computer science community or the general public.
%You should also justify the difficult of the project.
%As a rough guideline, your project should be either narrow but deep in a subfield of CS, or broadly reaching across subfields without being too shallow.
%It should be comparable to the amount of work/content in an upper-level elective.

\begin{comment}
    Why the project is interesting.
    To me:
        - want to gain experience in ML, particularly in setting up a real-world problem (as opposed to problems specifically engineered for textbooks)
        - video games are one of my hobbies, and I have seen the various ways sources of video game-related text can arise (reviews, descriptions, forums, etc.) - use NLP to inform ML model
    To general public:
        - recommender systems are prevalent in today's highly technological society. (Buying things on Amazon, watching content on Netflix, connecting with people on Facebook, professional development on LinkedIn, etc. - all use recommendation systems)
        - supervised learning is becoming part of recommender system infrastructure (particularly for content-based, rather than collaborative filtering-based, aspects)

    Difficulty of project:
    - ML - deciding, obtaining, and preprocessing data in an ethical, efficient way. This is good practice for the real world
    - iterative ML algorithm improvement - how to identify breakpoints and be skeptical of results, then make iterative changes to algorithm to fix results
\end{comment}

\section{Technical Background}

%This section introduces the technical knowledge necessary to understand your project, including any terminology and algorithms.
%You should assume that the reader is a CS undergraduate like yourself, but not necessarily familiar with AI/ML/HCI/apps/video games/etc.

\begin{comment}
    Recommendation systems.
        Content based
        Collaborative filtering
        Some RS applying both / extension of work (e.g., ant colony optimization paper?)

    Supervised ML.
        Machine learning
        Supervised learning
        SVM algorithm

    Relevant NLP content
        TF-IDF
        Word embeddings
\end{comment}

\section{Prior Work}

%This section describes of related and/or existing work.
%This could be scientific or scholarly, but may also be a survey of existing products/games.
%The goal of this section is to put your project in the context of what has already been done.

\begin{comment}
    Products.

        Collaborative filtering - Quantic Foundry, Steam Recommender

    Papers.

        Collaborative filtering
            Choi et al. https://library.ucsd.edu/dc/object/bb5021836n/_3_1.pdf

            Okon et al. https://www.researchgate.net/profile/Emmanuel-Okon/publication/324897120_An_Improved_Online_Book_Recommender_System_using_Collaborative_Filtering_Algorithm/links/5b25ee34a6fdcc697469689a/An-Improved-Online-Book-Recommender-System-using-Collaborative-Filtering-Algorithm.pdf

            Gohari et al. - CF + trust https://link.springer.com/article/10.1007/s10489-016-0830-y

        Content based

            Ryan et al. https://eis.ucsc.edu/papers/ryanEtAl_PeopleTendToLikeRelatedGames.pdf

            Vu and Bezemer - https://asgaard.ece.ualberta.ca/wp-content/uploads/2021/04/quang_fdg2021.pdf

        Combination of above two methods

            Perez-Marcos et al. https://link.springer.com/article/10.1007/s12652-020-01681-0

        Supervised ML and its role in recommendation

            ML in RS

                Nawrocka et al. https://ieeexplore.ieee.org/stamp/stamp.jsp?arnumber=8399650&tag=1

                Xu and Araki - SVM for personalized recommendations - https://ieeexplore.ieee.org/stamp/stamp.jsp?arnumber=1651358

                Using recommendation for sentiment analysis via ML. Anqi Zhang - https://openprairie.sdstate.edu/cgi/viewcontent.cgi?article=1487&context=etd2

                Examination of game reviews for game recs
                Meidl and Lytinen - https://ojs.aaai.org/index.php/AIIDE/article/view/12752/12600
                (read this more closely)

            NLP in RS
            TODO search "nlp for recommender system"

                Guo et al. https://dl.acm.org/doi/pdf/10.1145/3292500.3332290
                

            TODO search "word embeddings recommender system"

            TODO get tf-idf prior work
            
    

        See Lit Review Google document.
    
\end{comment}

\section{Methods}

%This section describes what exactly you will be working on.
%What are you building? How will it combine/incorporate ideas from the literature? Be specific about what you will be doing: talk about the specific algorithm you will implement/use, the specific dataset/platform/API, and what the outcome of your project will look like.
%All of these decisions should be justified as well.

\section{Evaluation Metrics}

%This section describes how you will evaluate your project.
%What will you be measuring, and how will you measure it?
%You might think about what would result in an F, a C, or an A for comps.
%Alternately, think about what are the minimal requirements for passing the class, what you might do if you had more time and resources, and what the best case scenario would be if everything went swimmingly.

\section{Evaluation Results and Discussion}

%limitations

\section{Ethical Considerations}

%Are there any ethical concerns that might arise from your project?
%You might think about whether your project perpetuates societal inequity (or could be used by others to do so), whether the data/platforms you are using is collected with informed consent and free of bias, and whether you might be subject to technological solutionism instead of working support/better the public infrastructure.
%Include a discussion of how you plan to mitigate these issues in your project.

\section{Future Work, and Conclusion}

\section{Timeline}

%A timeline of major milestones, with specific items to be completed by specific dates/months.
%Note that this timeline must start over the summer; otherwise you will unlikely have enough time to complete a project of the expected scope.
%As part of the discussion around your timeline, talk about what you already know that would help you with the project, and what you expect to have to learn to be successful.
%Include programming languages, technical concepts, as well as processes (e.g., user testing).

\section{Code Documentation}

This section will demonstrate that you have thought through the basics of how your code will work. You should include a diagram of the overall data flow of your program, including what the inputs and outputs of each component will be, and how they will be represented.

\section{Appendices}

\printbibliography

\end{document}

